\chapter{XSL-FO}
\label{chap:xsl-fo}

\section{What is it?}

XLS-FO\cite{wikixslfo,w3schoolxslfo} is a formatting language for XML data. The general idea behind XLS-FO is that XML data is converted using XLST to transform the XML data into XLS-FO. The XLS-FO document is then passed to the FO processor, which converts it into something readable (PDF or PS). 

\section{Structure}

It starts with a XML declaration and a root element. The root element also declares the namespace for XLS-FO. In the root element you would have a layout-master-set element which holds one or more page templates. The simple-page-master element, which is encapsulated in the layout-master-select, contains a single page name and must have a unique master-name attribute. The page-sequence elements describe the page contents and the master-reference attribute refers to the simple-page-master template (which must be the same name). 

\lstset{
	language=xml,
	tabsize=3,
	caption=An example of XSL-FO,
	label=code:xslfo,
	frame=shadowbox,
	rulesepcolor=\color{gray},
	xleftmargin=20pt,
	framexleftmargin=15pt,
	keywordstyle=\color{blue}\bf,
	commentstyle=\color{OliveGreen},
	stringstyle=\color{red},
	numbers=left,
	numberstyle=\tiny,
	numbersep=5pt,
	breaklines=true,
	showstringspaces=false,
	basicstyle=\footnotesize,
	emph={food,name,price},
	emphstyle={\color{magenta}}
}
\lstinputlisting{Chapters/02_XSL-FO.example1.xml}

As you can see in listing \ref{code:xslfo} the page-sequence refers to the simple-page-master, since it has the same name. The block is encapsulated in the flow, which is encapsulated in the page-sequence. The page-sequence refers to the simple-page-master template.